\NeedsTeXFormat{LaTeX2e}
% \documentclass[a4paper,12pt,onecolumn,oneside]{book}
\documentclass[12pt]{article}
\usepackage[unicode,colorlinks]{hyperref}
\usepackage{html}
\hypersetup{unicode=true} % Makes bookmarks work in Russian

% подавление висячих строк
\clubpenalty=10000
\widowpenalty=10000

% убрать переносы слов
\pretolerance=10000

% межабзацный отступ
\parskip=1.2ex


\hypersetup
{
    pdfauthor={Yuri Timofeev tim4dev@gmail.com},
    pdftitle={Webacula v.5.5. Installation Manual},
    pdfkeywords={webacula, manual, install, bacula}
}


\title{
  \Huge{Webacula v.5.5}
  \begin{center}
   \large{Installation Manual}
  \end{center}
}
\author{
  \begin{small}
    Copyright 2007, 2008, 2009, 2010, 2011 Yuri Timofeev \htmladdnormallink{tim4dev@gmail.com}{mailto:tim4dev@gmail.com}
  \end{small}
}
%\date{Dec 2010}

\begin{document}
\maketitle

\tableofcontents
\newpage

\section{About this manual}
\label{About this manual}

The basic features of Webacula see in README file.

This manual should give you to install or upgrade Webacula installation.

If you find errors or typos please
\htmladdnormallink{send a bug report}{http://sourceforge.net/tracker/?group_id=201199}.

Thanks.



\section{System Requirements}
\label{System Requirements}

To check the installed system packages, run from command line:
\begin{verbatim}docs/check_system_requirements.php\end{verbatim}

\textbf{NOTE}. The successful execution of the script does not indicate that your system is fully ready to work with Webacula.


Webacula also requires:
\begin{itemize}
  \item Bacula 5.0 or later
  \item Zend Framework version 1.10.0 or later
  \item Zend Framework is built with object-oriented PHP 5 and requires PHP 5.2.4 or later with PDO extension active.
        \htmladdnormallink{Please see the system requirements appendix}{http://framework.zend.com/manual/en/requirements.html}
        for more detailed information.
  \item Apache and \texttt{mod\_rewrite}. Or equivalent web-server, for example, nginx and \texttt{ngx\_http\_rewrite\_module}
  \item Installed \texttt{php-gd} package
  \item Installed \htmladdnormallink{http://php.net/dom}{http://php.net/dom} for use the RSS feed
  \item Browser compatibility: all jQuery UI plugins are tested for IE 6.0+, Firefox 3+, Safari 3.1+, Opera 9.6+, Google Chrome
\end{itemize}



\section{Install}
\label{Install}



\subsection{Make directory tree}
\label{Install:Make directory tree}

Login as root and make directory \texttt{/var/www/webacula} (for example).
Copy Webacula distribution to this directory.

\htmladdnormallink{Download minimal Zend Framework package}{http://framework.zend.com/} and extract.
Copy the contents from directory \\
\texttt{ZendFramework-*-minimal/library/Zend} \\
to \\
\texttt{webacula/library/Zend}

\textbf{NOTE}. If you use the Zend Framework for multiple sites, then you can place it in a folder that is part of
your PHP include path. By doing this, you will have access to the Zend Framework components in all PHP scripts.

The tree which should turn out as a result :

\begin{verbatim}
/var/www/webacula/
|-- application
|   |-- controllers
|   |-- models
|   `-- views
...
|-- data
|   |-- cache
|   |-- session
|   `-- tmp
...
|-- docs
|-- install
|-- html
|-- languages
`-- library
    |-- MyClass
    `-- Zend (here is Zend Framework package)
        |-- Acl
        |-- Auth
        |-- Cache
       ...
\end{verbatim}


Some directory description:

\texttt{application/}    All source code. Should be available to reading for the Web-server and
                  no access through the client Web-browser.

\texttt{html/}    Public code. Should be available to reading for the Web-server and for the client Web-browser.

\texttt{data/}    \textbf{IMPORTANT}. This directory, subdirectory and files in it
                  must NOT be available to access through the client Web-browser.

\texttt{data/cache/}  Cache directory for Zend\_Cache. Should be available to writing the Web-server and
                  no access through the client Web-browser.

\texttt{data/session/}   Storage for PHP session. Should be available to writing the Web-server and
                  no access through the client Web-browser.

\texttt{data/tmp/}   This directory which will be saved the file, which contains a list of files to Job restore.
                  This directory and files in it should be available to read from the Bacula Director and
                  to writing from the Web-server. And no access through the client Web-browser.




\subsection{config.ini}
\label{Install:config.ini}

Specify the parameters to connect to the Catalog database, timezone and other in \texttt{application/config.ini}


\end{document}
